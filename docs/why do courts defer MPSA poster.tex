%%%%%%%%%%%%%%%%%%%%%%%%%%%%%%%%%%%%%%%%%
% Jacobs Landscape Poster
% LaTeX Template
% Version 1.0 (29/03/13)
%
% Created by:
% Computational Physics and Biophysics Group, Jacobs University
% https://teamwork.jacobs-university.de:8443/confluence/display/CoPandBiG/LaTeX+Poster
% 
% Further modified by:
% Nathaniel Johnston (nathaniel@njohnston.ca)
%
% This template has been downloaded from:
% http://www.LaTeXTemplates.com
%
% License:
% CC BY-NC-SA 3.0 (http://creativecommons.org/licenses/by-nc-sa/3.0/)
%
%%%%%%%%%%%%%%%%%%%%%%%%%%%%%%%%%%%%%%%%%

%----------------------------------------------------------------------------------------
%	PACKAGES AND OTHER DOCUMENT CONFIGURATIONS
%----------------------------------------------------------------------------------------

\documentclass[final]{beamer}

\ProvidesPackage{beamerthemeconfposter}
\usepackage{geometry}
\usepackage{setspace}
\usepackage{amsfonts, amsmath, amssymb}
\usepackage{dcolumn, multirow}
\usepackage{graphicx}
\usepackage{anysize, indentfirst, setspace}
\usepackage{caption, rotating}
\usepackage{booktabs}
\usepackage{graphicx}
\usepackage{xcolor}
\usepackage{hyperref}
\usepackage{amsmath}
\usepackage{amssymb}
\usepackage{hanging}
\usepackage{float}
\usepackage[scale=1.24]{beamerposter} % Use the beamerposter package for laying out the poster

\usetheme{confposter} % Use the confposter theme supplied with this template
\definecolor{bostonuniversityred}{rgb}{0.8, 0.0, 0.0}
\definecolor{crimsonglory}{rgb}{0.75, 0.0, 0.2}
\setbeamercolor{block title}{fg=crimsonglory,bg=white} % Colors of the block titles
\setbeamercolor{block body}{fg=black,bg=white} % Colors of the body of blocks
\setbeamercolor{block alerted title}{fg=white,bg=gray} % Colors of the highlighted block titles
\setbeamercolor{block alerted body}{fg=black,bg=white} % Colors of the body of highlighted blocks
% Many more colors are available for use in beamerthemeconfposter.sty

%-----------------------------------------------------------
% Define the column widths and overall poster size
% To set effective sepwid, onecolwid and twocolwid values, first choose how many columns you want and how much separation you want between columns
% In this template, the separation width chosen is 0.024 of the paper width and a 4-column layout
% onecolwid should therefore be (1-(# of columns+1)*sepwid)/# of columns e.g. (1-(4+1)*0.024)/4 = 0.22
% Set twocolwid to be (2*onecolwid)+sepwid = 0.464
% Set threecolwid to be (3*onecolwid)+2*sepwid = 0.708

\newlength{\sepwid}
\newlength{\onecolwid}
\newlength{\twocolwid}
\newlength{\threecolwid}
\setlength{\paperwidth}{48in} % A0 width: 46.8in
\setlength{\paperheight}{36in} % A0 height: 33.1in
\setlength{\sepwid}{0.024\paperwidth} % Separation width (white space) between columns
\setlength{\onecolwid}{0.22\paperwidth} % Width of one column
\setlength{\twocolwid}{0.464\paperwidth} % Width of two columns
\setlength{\threecolwid}{0.708\paperwidth} % Width of three columns
\setlength{\topmargin}{-0.5in} % Reduce the top margin size
%-----------------------------------------------------------

\usepackage{graphicx}  % Required for including images

\usepackage{booktabs} % Top and bottom rules for tables

%----------------------------------------------------------------------------------------
%	TITLE SECTION 
%----------------------------------------------------------------------------------------

\title{Why Do Courts Defer to Administrative Agency Judgment?} % Poster title

\author{Devin Judge-Lord} % Author(s)

\institute{Department of Political Science, University of Wisconsin-Madison} % Institution(s)

%----------------------------------------------------------------------------------------

\begin{document}

%\addtobeamertemplate{block end}{}{\vspace*{2ex}} % White space under blocks
%\addtobeamertemplate{block alerted end}{}{\vspace*{2ex}} % White space under highlighted (alert) blocks

\setlength{\belowcaptionskip}{2ex} % White space under figures
\setlength\belowdisplayshortskip{2ex} % White space under equations

\begin{frame}[t] % The whole poster is enclosed in one beamer frame

\begin{columns}[t] % The whole poster consists of three major columns, the second of which is split into two columns twice - the [t] option aligns each column's content to the top

\begin{column}{\sepwid}\end{column} % Empty spacer column

\begin{column}{\onecolwid} % The first column

%----------------------------------------------------------------------------------------
%	OBJECTIVES
%----------------------------------------------------------------------------------------





%----------------------------------------------------------------------------------------
%	INTRODUCTION
%----------------------------------------------------------------------------------------
\begin{alertblock}{Puzzle}
Rulemaking = Deference
  \begin{itemize}
    \item ``deference should be accorded to the agency's interpretation'' (\textit{Chevron v. NRDC} 1984) 
    \item ``APA notice and comment [rulemaking] is designed to assure due deliberation'' (\textit{US v. Mead Corp.} 2001)
  \end{itemize}
  Rulemaking $\neq$ Deference
    \begin{itemize}
      \item ``because there is more at stake'' (Elliott and Schuck 1990, Richards et al. 2006)
  
    \end{itemize}

\end{alertblock}


\begin{block}{Abstract}

I investigate the conventional wisdom that courts are more likely to uphold federal agency policies made through notice-and-comment rulemaking. I find that Supreme Court Justices have different preferences for deference, but overall, are actually less likely to show deference in rulemaking cases. The evidence does not suggest that this is due to justices voting more ideologically due to the greater stakes in rulemaking cases as implied by previous research. I then investigate potential selection effects at the circuit court and certiorari stages that could cause rulemaking cases to lose at the Supreme Court. The results offer no evidence that deference differs between circuit courts and the Supreme Court. While the Supreme Court may disproportionately opt to review rulemaking cases, it does not appear to do this in order to systematically discipline lower courts. Finally, I examine elements of the rulemaking process itself that may affect judicial review. Greater participation in rulemaking correlates with deference, as does an agency's choice to regulate less rather than more. The puzzling negative relationship between rulemaking and deference remains unresolved, but, contrary to previous research, preliminary results suggest that justices may defer more to agency rulemaking on more salient issues where stakes are highest. %The factors that the Court deems important likely affect agency policymaking. In this sense, defining doctrines of deference is one way that justices shape policy. 

\end{block}



%------------------------------------------------

\begin{block}{Data and Methods}
Data
\begin{itemize}
	\item All Supreme Court cases and a random sample of 175 federal circuit court cases 1984-2000 that mention ``Administrative Law'' in the LexisNexis headnotes.
	\begin{itemize}
		\item Conservative-Liberal Ideology: Justices = Segal-Cover scores (-1:1), Policy = Spaeth criteria (-1, 0, 1)
		\item Parties, briefs, opinions, outcomes
	\end{itemize}
	\item Final Rule documents from federal register archives. 
	\begin{itemize}
		\item Notice of Proposed Rulemaking (NPRM)
		\item Comments filed during notice-and-comment rulemaking
	\end{itemize}
\end{itemize}


Logistic regression with two-tailed tests.\\

\begin{table}
\begin{tabular}{@{\extracolsep{5pt}}ccccccc} 
& &  & &  \\
& & &\multicolumn{3}{c}{H2(disciplining)}\\
& & & & & \\
&\fbox{Policymaking} & $\rightarrow$ & \fbox{Circuit Courts} &$\rightarrow$ & \fbox{Supreme Court}\\
& & & & \\
& H3.1(expertise) &  &  & & H1(stakes)\\
& H3.2(accountability) &   & & & H1.1(agreement)\\
& H3.3(participation) & & & \\
\end{tabular}
\end{table}

\end{block}



%----------------------------------------------------------------------------------------

\end{column} % End of the first column

%\begin{column}{\sepwid}\end{column} % Empty spacer column

%\begin{column}{\twocolwid} % Begin a column which is two columns wide (column 2)

%\begin{columns}[t,totalwidth=\twocolwid] % Split up the two columns wide column

%\begin{column}{\onecolwid}\vspace{-.6in} % The first column within column 2 (column 2.1)

%----------------------------------------------------------------------------------------
%	MATERIALS
%----------------------------------------------------------------------------------------



%----------------------------------------------------------------------------------------

%\end{column} % End of column 2.1

%\begin{column}{\onecolwid}\vspace{-.6in} % The second column within column 2 (column 2.2)

%----------------------------------------------------------------------------------------
%	METHODS
%----------------------------------------------------------------------------------------



%----------------------------------------------------------------------------------------

%\end{column} % End of column 2.2

%\end{columns} % End of the split of column 2 - any content after this will now take up 2 columns width

%----------------------------------------------------------------------------------------
%	IMPORTANT RESULT
%----------------------------------------------------------------------------------------

%\begin{alertblock}{Important Result}

%Lorem ipsum dolor \textbf{sit amet}, consectetur adipiscing elit. Sed commodo molestie %porta. Sed ultrices scelerisque sapien ac commodo. Donec ut volutpat elit.

%\end{alertblock} 

%----------------------------------------------------------------------------------------

%\begin{columns}[t,totalwidth=\twocolwid] % Split up the two columns wide column again

\begin{column}{\onecolwid} % The first column within column 2 (column 2.1)
\begin{alertblock}{Question}
What makes bureaucratic policy legitimate?

Specifically, why do courts uphold agency policies?\begin{enumerate}
\item How justices decide
\item How policies are challenged and selected for review
\item How policy is made
%\item Euismod non erat. Nam ultricies pellentesque nunc, ultrices volutpat nisl ultrices a.
\end{enumerate}

\end{alertblock}




%----------------------------------------------------------------------------------------

%\end{column} % End of column 2.1

%\begin{column}{\onecolwid} % The second column within column 2 (column 2.2)

%----------------------------------------------------------------------------------------
%	RESULTS
%----------------------------------------------------------------------------------------

\begin{block}{Results}

$\boxtimes$ 
H1: Rulemaking cases lose because stakes are high\\
$\boxtimes$ H1.1: Justices vote more ideologically in rulemaking cases
\begin{figure}
\caption{1: Votes for Agency in Administrative Law Cases (Logit Coefficients)}
\includegraphics[width=24cm]{UW_main_white.pdf}
\centering
\end{figure}
\footnotesize
\color{red} *All administrative law cases. ``Rulemaking'' = Rule-like policymaking cases\\
\color{black} *Rule-like policymaking cases. ``Rulemaking'' = Notice-and-comment rulemaking cases

\end{block}

\begin{block}{}

$\boxtimes$ H2: SCOTUS disciplines lower courts for being too deferential\begin{itemize}
\item $\boxtimes$ H2.1: Circuit courts are more deferential to rulemaking
\item $\boxtimes$ H2.2: Circuit courts defer in different kinds of cases
\end{itemize}
\begin{table}[H] \centering \footnotesize
  \caption{1: Votes for Agency in Circuit Court Administrative Law Cases} 
  \label{} 
\begin{tabular}{@{\extracolsep{5pt}}lc} 
\\[-1.8ex]\hline 
\hline \\[-1.8ex] 
 & \multicolumn{1}{c}{\textit{Dependent variable:}} \\ 
\cline{2-2} 
\\[-1.8ex] & Deference Vote \\ 
\hline \\[-1.8ex] 
 Rulemaking & $-$1.429$^{***}$ \\ 
  & (0.465) \\ 
  & \\ 
 Pro-Deference Briefs & 2.250$^{***}$ \\ 
  & (0.613) \\ 
  & \\ 
 Anti-Deference Briefs & 0.274$^{**}$ \\ 
  & (0.118) \\ 
  & \\ 
 Ideology*Agency Policy & $-$2.156$^{***}$ \\ 
  & (0.727) \\ 
  & \\ 
 Ideology*Policy*Rulemaking & $-$0.767 \\ 
  & (1.507) \\ 
  & \\ 
 Constant & $-$1.122 \\ 
  & (0.784) \\ 
  & \\ 
\hline \\[-1.8ex] 
Observations & 410 \\ 
Log Likelihood & $-$246.870 \\ 
Akaike Inf. Crit. & 527.739 \\ 
\hline 
\hline \\[-1.8ex] 
\textit{(See Paper for Full Model)}& \multicolumn{1}{r}{$^{*}$p$<$0.1; $^{**}$p$<$0.05; $^{***}$p$<$0.01} \\
\end{tabular} 
\end{table} 


\end{block}



%----------------------------------------------------------------------------------------

\end{column} % End of column 2.2

%\end{columns} % End of the split of column 2

%\end{column} % End of the second column

\begin{column}{\sepwid}\end{column} % Empty spacer column

\begin{column}{\onecolwid} % The third column

%----------------------------------------------------------------------------------------
%	CONCLUSION
%----------------------------------------------------------------------------------------
\begin{block}{}
$\text{\rlap{$\checkmark$}}\square$
H3: The process by which policy is made affects judicial review\begin{itemize}
	\item $\boxtimes$? H3.1: when requiring special expertise
	\item $\boxtimes$? H3.2: when accountable to political principals
	\item $\text{\rlap{$\checkmark$}}\square$? H3.3: when participatory
\end{itemize}
\begin{table}[!b] \centering \tiny
  \caption{3: Expectations for Policymaking Variables} 
  \label{} 
\begin{tabular}{@{\extracolsep{5pt}}lcccc} 
\\[-1.8ex]\hline 
\hline \\[-1.8ex] 
 & \multicolumn{4}{c}{If deference is due to:} \\ 
\cline{2-5} 
\\[-1.8ex] & Ideology & \multicolumn{3}{c}{Policymaking Process} \\ 
\cline{2-5} 
\\[-1.8ex] & Policy Agreement & Expertise & Accountability & Participation\\ 
\hline \\[-1.8ex] 
Ideology*Policy & + &  & &\\ 
  & & & &\\ 
Pres. Can Fire Head &  &  & + &\\ 
  & & & &\\ 
Statue Length &  & ? & ? &\\ 
  & & & &\\ 
Rulemaking-like Policy &  & + &  &\\ 
  & & & &\\ 
Rulemaking (NPRM) & & + & + & +\\ 
  & & & &\\ 
Rule Complexity &  & + & -? & -\\ 
  & & & &\\ 
Special Expertise &  & + & -? & -\\ 
  & & & &\\ 
Conflicting Statutes &  & + & ? & \\ 
  & & & &\\ 
Justices Mention Legislative History &  &  & ? & \\ 
  & & & &\\ 
Justices Mention APA &  & ? & ? & ?\\ 
  & & & &\\ 
Comment Deadline Extended &  &  & + & +\\ 
  & & & &\\ 
Number of Comments &  & + & + & +\\ 
  & & & &\\ 
More Pro-Regulation Comments &  &  & +? & +\\ 
  & & & &\\ 
Agency Overreach &  &  & ? & \\ 
  & & & &\\ 
Bad Faith &  & - & - & - \\ 
  & & & &\\ 
\hline \\[-1.8ex]   
\end{tabular} 
\end{table}  

\end{block}

\begin{block}{}

\begin{figure}
\caption{2: Votes for Agency in Rulemaking Cases (Logit Coefficients)}
\includegraphics[width=28cm]{UW_main_white.pdf}
\centering
\end{figure}

\footnotesize
\color{blue} *Notice-and-comment rulemaking cases only.\\

\begin{figure}
\caption{3: Votes for Agency in Policymaking Cases (Logit Coefficients)}
\includegraphics[width=26cm]{UW_main_white.pdf}
\centering
\end{figure}

\footnotesize
\color{black} *Rule-like policymaking cases. ``Rulemaking'' = Notice-and-comment rulemaking.\\

\end{block}

\end{column}

\begin{column}{\onecolwid}  % fourth col

%\begin{block}{Conclusion}

%Nunc tempus venenatis facilisis. \textbf{Curabitur suscipit} consequat eros non porttitor. Sed a massa dolor, id ornare enim. Fusce quis massa dictum tortor \textbf{tincidunt mattis}. Donec quam est, lobortis quis pretium at, laoreet scelerisque lacus. Nam quis odio enim, in molestie libero. Vivamus cursus mi at \textit{nulla elementum sollicitudin}.

%\end{block}

%----------------------------------------------------------------------------------------
%	ADDITIONAL INFORMATION
%----------------------------------------------------------------------------------------

\begin{block}{Conclusion: The Puzzle Persists}
Why are agencies more likely to lose when they use rulemaking?
\begin{itemize}
\item Not the high stakes
      \begin{itemize}
        \item Not more ideological
        \item Less deference than for policies with similar stakes
      \end{itemize}
\item Not disciplining circuit courts
      \begin{itemize}
        \item No difference in deference or ideological voting
      \end{itemize}
\item Policy process may matter, but in unexpected ways
      \begin{itemize}
        \item More controversial rules get more deference
      \end{itemize}
\item Salience and Administrative Procedures Act compliance?
      \begin{itemize}
        \item Amicus briefs, rule comments = deference
        \item APA = procedural liabilities
      \end{itemize}
\end{itemize}

\begin{figure}
\caption{4: Principal Components Analysis, Supreme Court Rulemaking Cases}
\includegraphics[width=25cm]{UW_main_white.pdf}
\centering
\end{figure}



The process by which policies are made affects how courts treat them. Yet, observed patterns are not entirely consistent with theories of legitimacy in the literature on bureaucratic policymaking. I identify new factors that predict judicial deference, including the number of comments a policy received and whether it is challenged as doing more or less than Congress intended. I also suggest new interpretations of previously identified variables.  

\end{block}

%----------------------------------------------------------------------------------------
%	REFERENCES
%----------------------------------------------------------------------------------------

\begin{block}{References}
\begin{hangparas}{.5in}{1}
\small
Bailey, Michael A., and Forrest Maltzman. 2008. ``Does Legal Doctrine Matter? Unpacking Law and Policy Preferences on the U.S. Supreme Court." \textit{American Political Science Review} 102(3): 369–84.

Elliot, Donald, and Peter Schuck. 1990. ``To the Chevron Station: An Empirical Study of Federal Administrative Law.'' \textit{Yale Law School Faculty Scholarship Series.} Paper 2204.

Richards, Mark, Joeseph Smith, and Herbert Kritzer. 2006 ``Does \textit{Chevron} Matter?'' \textit{Law \& Policy} 28(4): 444-469.
\end{hangparas}
%\nocite{*} % Insert publications even if they are not cited in the poster
%\small{\bibliographystyle{unsrt}
%\bibliography{sample}\vspace{0.75in}}

\end{block}

%----------------------------------------------------------------------------------------
%	ACKNOWLEDGEMENTS
%----------------------------------------------------------------------------------------

%\setbeamercolor{block title}{fg=crimsonglory,bg=white} % Change the block title color

%\begin{block}{Acknowledgements}

%\small{\rmfamily{Nam mollis tristique neque eu luctus. Suspendisse rutrum congue nisi sed convallis. Aenean id neque dolor. Pellentesque habitant morbi tristique senectus et netus et malesuada fames ac turpis egestas.}} \\

%\end{block}

%----------------------------------------------------------------------------------------
%	CONTACT INFORMATION
%----------------------------------------------------------------------------------------

\setbeamercolor{block alerted title}{fg=black,bg=gray} % Change the alert block title colors
\setbeamercolor{block alerted body}{fg=black,bg=white} % Change the alert block body colors

%\begin{alertblock}{Contact Information}

%\begin{itemize}
%\item Web: \href{http://www.university.edu/smithlab}{http://www.university.edu/smithlab}
%\item Email: \href{mailto:john@smith.com}{john@smith.com}
%\item Phone: +1 (000) 111 1111
%\end{itemize}

%\end{alertblock}

\begin{center}
\begin{tabular}{ccc}
\includegraphics[width=.5\linewidth]{UW_main_white.pdf} & \hfill & 
\end{tabular}\\
JudgeLord@wisc.edu
\end{center}


%----------------------------------------------------------------------------------------

\end{column} % End of the third column

\end{columns} % End of all the columns in the poster

\end{frame} % End of the enclosing frame

\end{document}
